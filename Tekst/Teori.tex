\subsection{Fysik teori}
Teoretisk udledning af formler til det skrå kast\\
Begyndelseshastigheden for et kast er givet ud fra vektoren $v_{0}$ og vinklen a, det betyder altså at hastighedsvektorens komposanter kan findes ved:\\

\begin{center}
\begin{math}
\centering
v_{x} = v_{0} \cdot cos(a) \quad \textrm{og} \quad v_{y} = v_{0} \cdot sin(a)
\end{math}
\end{center}

Samtidigt vides det at x- og y-slut kan skrives som to bevægelsesligninger:\\

\begin{center}
\begin{math}
\centering
y = -\dfrac{1}{2} \cdot g \cdot t^{2} + v_{y} \cdot t + y_{0}
\end{math}
\end{center}



Hvor dette er formlen for strækning inden for faget kinematik, omskrevet til at passe på y-aksen i det skrå kast. Her skiftes strækningen ud med y og accelerationen med g, da det er tyngdeaccelerationen som kastet arbejder imod:\\

\begin{center}
\begin{math}
\centering
s = -\dfrac{1}{2} \cdot g \cdot t^{2} + v_{0} \cdot t + s_{0} 
\end{math}
\end{center}

x er derved givet ved konstant hastighed, denne får altså ligningen:\\

\begin{center}
\begin{math}
\centering
x = v_{x} \cdot t
\end{math}
\end{center}

Hvor denne er omskrevet fra formlen for konstant hastighed:\\

\begin{center}
\begin{math}
\centering
v = \dfrac{\Delta s}{\Delta t}
\end{math}
\end{center}




Hvor der isoleres for strækningen:\\

\begin{center}
\begin{math}
v = \dfrac{\Delta s}{\Delta t} \longrightarrow \Delta s = v \cdot \delta t
\end{math}
\end{center}






















