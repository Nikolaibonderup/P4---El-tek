%Dokument til kravspecifikationer
\section{Kravspecifikationer}

De opstillede krav fra el-tek:\\

Der skal bruges en “interrupt” (HW - Kontakt, SW - Timer).
\begin{itemize}
\item Der skal altså i projektet bruges enten en kontakt eller timer i vores kredsløb. Det er påkrævet at denne har en relevant betydning for selve kredsløbet og ikke har en meningsløs funktion.
\end{itemize}



Der skal være et sensor input: analog til digital konvertering.
\begin{itemize}
\item Dette forstås som at der skal bruges en type sensor, som måler noget analogt, der derefter kan oversættes til noget digitalt vha. en mikroprocessor. Dette kunne for eksempel være en afstandssensor.
\end{itemize}

Digital til Analog konvertering.
\begin{itemize}
\item Dette vil ved brug af Arduino i de fleste tilfælde være at bruge et PWM signal til kontrol af et elektronisk element.
\end{itemize}


Der skal bruges datakommunikation (til pc, viserinstrument eller trådløst element).
\begin{itemize}
\item Dette ville være en form for input/output type af kontrol i forhold til vores produkt. Her skal gruppen kunne give en eller anden form for ordre til produktet og produktet skal så udføre en bestemt handling. Dette kunne opfyldes ved at styre kanonens vinkel med en controller vha. bluetooth.
\end{itemize}

Et print til en mikrocontroller.
\begin{itemize}
\item Der skal til produktet bruges et selvproduceret mikrocontrollerprint. Denne microcontroller skal kunne styre en af hovedelementerne i selve produktet for at opfylde kravet om at have en relevant funktion.
\end{itemize}

Udover de specifikke krav skal der også indrages relevant fysikteoretisk arbejde, som udmunder i afleveringen af videnskabelig dokumentation vedrørende det valgte fysikteori og el-tek produkt.
