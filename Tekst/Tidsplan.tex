%%Tidsplan
\subsection{Tidsplan}

I dette projekt har vi fået en bestemt mængde af tid med vejledning på. Der er så senere på grund af mangel på tid brugt timer i værkstedet uden vejledning hvor gruppen helt selvstændigt har arbejdet med produktudvikling. Herunder ses en tegnforklaring til opsætningen af de viste tidsplaner.



\begin{figure}[H]
\centering
\includegraphics[scale=0.5]{Billeder/Tidsplan_symboler}
\caption{Symboler som fremkommer i tidsplanen.}
\label{fig:tidsplan_symboler}
\end{figure}


\begin{itemize}

\item “Planlagt” bliver sat på de felter hvor en opgave er planlagt.

\item “Udført” sættes i alle de felter hvor en opgave er løst, ligegyldigt om det var planlagt.

\item “Elevtid/Værksted” er timer som ikke er skemalagte, men hvor gruppen stadig har arbejdet i el-værkstedet.

\item “Aflevering” er sat på de datoer hvor et stykke arbejde skal afleveres. Et “afleveringsfelt” markeres med “Indleveret” for at vise at en aflevering er blevet indleveret. 

\end{itemize}


I bunden er de tiltænkte timer indskrevet som skal bruges på projektet. Herunder ses “Speciel Hændelse” som er markeret på tidsplanerne, hvis sådanne finder sted.




\begin{figure}[H]
\centering
\includegraphics[scale=0.8, angle = 90]{Billeder/Tidsplan_1}
\caption{Endelige udkast af tidsplanen.}
\label{fig:tidsplan_1}
\end{figure}